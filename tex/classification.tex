\documentclass[11pt]{article}

\usepackage{amsmath}
\usepackage{amsthm}
\usepackage{amssymb}
\usepackage{algorithm}
\usepackage{algorithmic}

\usepackage{graphicx}
\graphicspath{ {figures/} }

\usepackage{subcaption}

\newtheorem{theorem}{Theorem}[section]
\newtheorem{lemma}{Lemma}[section]
\newtheorem{corollary}{Corollary}[section]
\newtheorem{proposition}{Proposition}[section]
\newtheorem*{claim}{Claim}

\theoremstyle{definition}
\newtheorem{definition}{Definition}[section]
\newtheorem{example}{Example}[section]

\newcommand{\R}{\mathbb{R}}
\newcommand{\Q}{\mathbb{Q}}
\newcommand{\vect}[1]{\mathbf{#1}}

\title{Wide-Reach Classification}
\author{Vincenzo Liberatore}

\begin{document}

\maketitle

Let $P$ be a finite multiset of integer {\em positive samples\/} and 
$N$ be a finite multiset of integer {\em negative samples\/} in $\R^n$.
Given a multiset $X$, we denote as $\nu_X(\vect{x})$ the multiplicity of $\vect{x}$ in $X$. 
Given a hyperplane $h = (\vect{w}, c)$ in $\R^n$ 
specified by a normal vector $\vect{w} = (w_1 ~ w_2 \dots ~ w_n)^T \in \Q^n$ and a bias $c \in \Q$,
define $C(P, h) = \{ \vect{x} \in P : \vect{x}^T \vect{w} > c \}$
to be the set of positive samples that are {\em consistent with the hyperplane\/} $(\vect{w}, c)$,
and similarly $C(N, h) = \{ \vect{x} \in P : \vect{x}^T \vect{w} < c \}$
to be the set of negative samples that are {\em consistent with the hyperplane\/} $(\vect{w}, c)$.
A hyperplane $h$ will be said to {\em satisfy a sample\/} 
$\vect{x} \in P$ ($\vect{x} \in N$)
if $\vect{x} \in C(P, h)$ ($\vect{x} \in C(N, h)$)
The {\em Maximum Hyperplane Consistency\/} problem is the problem in which,
given $P$ and $N$, we wish to find a hyperplane $h$
to maximize $|C(P, h)| + |C(N, h)|$, the number of samples 
that are consistent with the given hyperplane \cite{countingapprox}.

\iffalse
The special case in which $c$ is required to be $c = 0$ is the {\em Open Hemisphere\/} problem,
which is known to be NP-complete in general but solvable in polynomial time
when $n$ is constant \cite{openhemisphere}.
In the Open Hemisphere problem, it can be assumed without loss of generality that 
\begin{itemize} 
\item $N = \emptyset$ because 
any negative sample $\vect{x}$ can be replaced by the positive sample $-\vect{x}$
\item The zero vector $\vect{0}$ is not a sample because it cannot be satisfied by any hyperplane
\end{itemize}

The Maximum Hyperplane Consistency problem can be approximated within a factor of 2 \cite{mhc}.
The main idea of the algorithm is to process one dimension at a time 
by setting the component of $w$ along that dimension to maximize the number of satisfied vectors
whose only non-zero component is along that dimension.

Before discussing the algorithm, 
consider the simple problem in which we are given $m$ rationals $r_i$ and 
$s_i \in \{ \pm 1 \}$ coefficients ($i = 1, 2, \dots, m$),
and we wish to find a rational number $w$ that satisfies as many as possible of the inequalities
$x_i w > r_i$.
Let $r_i' = r_i / x_i$ and assume without loss of generality that 
$r_1' \leq r_2' \leq \dots \leq r_m'$.
Observe that the number of satisfied inequalities is constant in the 
intervals of $w$ corresponding to
$(-infty, r_1'), (r_1', r_2'), \dots (r_m', \infty)$.
Then, the algorithm calculates the number of satisfied inequalities in each of these intervals 
as well as in the points $r_1', r_2', \dots, r_m'$, and pick the
value of $w$ that maximizes the number of satisfied inequalities. 

Returning to the Maximum Hyperplane Consistency problem,  
the algorithm starts with the special case of the zero sample $\vect{0}$, in which 
case it sets the bias $c$ to maximize the number of copies of $\vect{0}$ that are satisfied by
the assignment 
$$c = \begin{cases} 
1 & \text{if} \nu_N(\vect{0}) > \nu_P(\vect{0})\\
-1 & \text{if} $\nu_P(\vect{0}) > \nu_C(\vect{0})$\\
0 & \text{otherwise}
\end{cases}
At this point, the algorithm proceeds as a sequence of iterations to set the value of $\vect{w}$.
Assume that, at the beginning of the current iteration,
the algorithm has set the value of $c, w_{i+1}, \dots, w_n$.
Let $r(\vect{x}) = c - \sum_{j = i + 1}^n w_j x_j$ for any sample $\vect{x}$.
The algorithm checks if there is any sample that has a single non-zero component among its first $i$
(in other words, given a sample $\vect{x} = (x_1 ~ x_2 \dots x_i ~ x_{i+1} ~ x_n)^T$,
we search for a sample in which only one of $x_1, x_2, \dots , x_i$ is non zero).
If there is no such sample, the algorithm sets $w_i = 0$ and moves on to the next iteration.
Otherwise, assume without loss of generality that the non-zero component is the $i$th,
and let $S_i$ be the multiset of samples that are zero save in the $i$th component. 
The algorithm sets $w_i$ so as to maximize the number of samples in $S_i$ that are satisfied. 
Specifically, it considers the inequalities 
$x_i w_i > r(\vect{x})$ for $\vect{x} \in P$ and 
$x_i w_i < r(\vect{x})$ for $\vect{x} \in N$, and
chooses any value of $w_i$ that maximizes the number of such inequalities that are satisfied. 
It then proceeds to the next iteration.
To justify the correctness, note that a sample $\vect{x}$ is considered at only one iteration or,
in other words, that the algorithm partitions the samples into the $S_i$s.
Furthermore, the satisfied samples in $S_i$ are never less than the once that are not satisfied,
which means that the algorithm satisfies at least half of the samples.


their $i$th component.
The algorithm sets the value of $w_i$ so that as many samples in $S$ as possible can be satisfied.


whose only non-zero component is along the $i$th dimension.
The algorithm sets the value of $w_n$ so that as many $S$ samples as possible can be satisfied,
and checks again.  

If there is such a sample $\vect{x}$, assume that $\vect{x}$ lies 


maximize the number of samples that are satisfied by  

Initially, the residuals are identically zero.
Then, the algorithm sets the bias $c$ to be maximize the number of copies of the zero vector 
$\vect{0}$ that are satisfied, and specifically,
$$c = \begin{cases} 
1 & \text{if} \nu_N(\vect{0}) > \nu_P(\vect{0})\\
-1 & \text{if} $\nu_P(\vect{0}) > \nu_C(\vect{0})$\\
0 & \text{otherwise}
\end{cases}
All copies of the $\vect{0}$ sample are removed.
The algorithm maintains a {\em residual\/} $r(\vect{x})$ for each remaining sample $\vect{x}$,
and the residuals are initialized to $c$.
Then, the algorithm specifies incrementally each component of $\vect{w}$. 
To set the next component of $\vect{w}$, the algorithm checks if there is any sample
that has a single non-zero component. 
If there is no such sample, the algorithm sets $w_n = 0$ and solves the 
problem recursively in $\R^{n-1}$ disregarding the samples' last component 
and starting from the given residuals.
If there is such a sample $\vect{x}$, 
assume without loss of generality that its $n$th component $x_n \neq 0$ is the only non-zero one.
Let $S'$ be the set of samples that are non-zero only in their $n$th component.
The algorithm seeks to maximize the number of samples in $S'$ that can be satisfied by setting
$w_n$.  

The residuals are then updated to $c$ for all positive samples and to $-c$ for all negative samples.


At this point, the negative samples $\vect{x}$ are replaced by positive samples $- \vect{x}$.

Then, the algorithm specifies incrementally each component of $\vect{w}$. 
To set the next component of $\vect{w}$, the algorithm checks if there is any sample
that has a single non-zero component. 
If there is no such sample, the algorithm sets $w_n = 0$ and solves the 
problem recursively in $\R^{n-1}$ disregarding the samples' last component with the given
residuals.
If there is such a sample $\vect{x}$, 
assume without loss of generality that its $n$th component $x_n \neq 0$ is the only non-zero one.


First, the algorithm sets $c = 0$ and replaces every negative sample $\vect{x} \in N$
by the positive sample $- \vect{x}$. 
Then, the algorithm specifies incrementally each component of $\vect{w}$. 
To set the next component of $\vect{w}$, the algorithm checks if there is any sample
that has a single non-zero component. 
If there is no such sample, the algorithm sets $w_n = 0$ and solves recursively the 
Maximum Hyperplane Consistency problem in $\R^{n-1}$ disregarding the samples' last component.
If there is such a sample $\vect{x}$, 
assume without loss of generality that its $n$th component $x_n \neq 0$ is the only non-zero one.
Let $P'$ be the set of samples that are non-zero only in their $n$th component.
The algorithm sets $w_n = 1$ if $P'$ contains more non-negative samples than non-positive ones,
and $w_n = -1$ otherwise.
Then, the algorithm solves recursively the Maximum Hyperplane Consistency problem in $\R^{n-1}$ 
over $P - P'$ disregarding the samples' last component.
To prove the approximatio ratio, assume by induction hypothesis that in the recursive call
the algorithm satisfies at least one half of its samples. 
Then, the algorithm satisfies at least $|P| / 2$ samples in the first case,
and $|P'| / 2 + |P - P'| / 2 = |P| / 2$ samples in the second case, which proves the claim.
Since tha algorithm always sets $c = 0$, it solves both the Maximum Hyperplane Consistency problem
and the Open Hemishpere problem.

A randomized algorithm for both problem is simple and achieves an approximation ratio of 2. 
Set $c = 0$ and choose a random vector $\vect{w}$ on the unit hypershpere, say, with the
method in \cite{rndvect}. Then, for any (positive) sample $\vect{x} \in P$,
$\Pr [ \vect{w}^T \vect{x} > 0 ] = 1 / 2$.

\end{document}
Semi-definite programming.
Epsilon.




\fi

\bibliographystyle{plain}
\bibliography{classification}

\end{document}
  
